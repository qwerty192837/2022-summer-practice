\documentclass[t]{beamer}
\usecolortheme{beaver} 
\useinnertheme{circles}
\useinnertheme{rectangles}

%%% Работа с русским языком
\usepackage{cmap}					% поиск в PDF
\usepackage{mathtext} 				% русские буквы в формулах
\usepackage[T2A]{fontenc}			% кодировка
\usepackage[utf8]{inputenc}			% кодировка исходного текста
\usepackage[english,russian]{babel}	% локализация и переносы

%%% Дополнительная работа с математикой
\usepackage{amsmath,amsfonts,amssymb,amsthm,mathtools} % AMS
\usepackage{icomma} % "Умная" запятая: $0,2$ --- число, $0, 2$ --- перечисление

%% Свои команды
\DeclareMathOperator{\sgn}{\mathop{sgn}}

%% Перенос знаков в формулах (по Львовскому)
\newcommand*{\hm}[1]{#1\nobreak\discretionary{}
{\hbox{$\mathsurround=0pt #1$}}{}}

%%% Другие пакеты
\usepackage{lastpage} % Узнать, сколько всего страниц в документе.
\usepackage{soul} % Модификаторы начертания
\usepackage{csquotes} % Еще инструменты для ссылок
%\usepackage[style=authoryear,maxcitenames=2,backend=biber,sorting=nty]{biblatex}
\usepackage{multicol} % Несколько колонок
\usepackage{listings}
\usepackage{relsize}
\usepackage{hyperref}


\title{Crossed Wires}

\author{Кулигин Данил и Хуснуллин Риналь}
\date{1 июля 2022 г.}
\institute[БФУ им. Иммануила Канта]{ИФМНиИТ \\ Компьютерная безопасность, 2 курс}

\begin{document}

\frame[plain]{\titlepage}	% Титульный слайд

\section{Условие задачи Crossed Wires}
\subsection{Описание с CryptoHack}
\begin{frame}
	\frametitle{\insertsection}
	\framesubtitle{\insertsubsection}
    \begin{itemize}
    "Я попросил своих друзей зашифровать наш секретный флаг, прежде чем отправлять его мне, но вместо того, чтобы использовать мой ключ, все они использовали свой собственный! Вы можете помочь?"\newline\newline
    
    Так же к задаче прилагаются файлы \slshape source.py \upshape и \slshape output.txt.
    source.py \upshape содержит код, при выполнении которого мы получим содержимое файла \slshape output.txt. \upshape Сам же \slshape output.txt \upshape содержит шифрованный флаг, секретный ключ (N, d) и публичные ключи ([(N,k1),(N,k2),...])

    \end{itemize}
\end{frame}

\section{Решение задачи Crossed Wires}

\subsection{Код решения данной задачи}

\begin{frame}[fragile]
	\frametitle{\insertsection} 
	\framesubtitle{\insertsubsection}
	\footnotesize
	\smaller
\begin{lstlisting}[language=Python]
from Crypto.Util.number import inverse, long_to_bytes

encrypted_flag = 203046102795781867381727662242247931...
e = 0x10001 # e = 65537

my_private_key = (217113..., 273441...)
# n = 217113...
# d = 273441...

friends_keys_list = [(217113..., 106979), (217113..., 108533), 
(217113..., 69557), (217113..., 97117), (217113..., 103231)]
# keys = [106979, 108533, 69557, 97117, 103231]

prod = 1
for key in friends_keys_list:
    prod *= key[1]
    
D = inverse(prod, e*my_private_key[1]-1)
print(long_to_bytes(pow(encrypted_flag,d,my_private_key[0])))

# Output: b'crypto{3ncrypt_y0ur_s3cr3t_w1th_y0ur_fr1end5_
publ1c_k3y}'
\end{lstlisting}
	
\end{frame}


\subsection{Обьяснение кода}

\begin{frame}
	\frametitle{Решение задачи Crossed Wires}
	\framesubtitle{\insertsubsection}
    \begin{enumerate}
        \item<1-> $\sqsupset k_1, k_2, k_3, k_4, k_5$ - ключи друзей (keys) и введём переменную $prod = k_1 \cdot k_2 \cdot ... \cdot k_5$
        \item<2-> Предположим, что $m$ - это флаг, тогда шифртекст $c$ вычисляется так: \( m^{prod} \pmod {n} = c \)
        \item<3-> Заметим, что не обязательно вычислять $\varphi(n)$ или делать факторизацию $n$. Достаточно посчитать: $e \cdot d - 1 = k \cdot \varphi(n)$ [ссылка 2]
        \item<4-> Тогда мы получим $D = prod^{-1} \pmod {e \cdot d - 1}$ и выведем в консоль $c^D \pmod{n}$, предварительно применив функцию long\_to\_bytes, чтобы привести сообщение в исходный вид.
    \end{enumerate}
\end{frame}

\section{Концепция RSA}
\subsection{Математическая модель}

\begin{frame}[t] % [t], [c], или [b] --- вертикальное выравнивание на слайде (верх, центр, низ)
	\frametitle{\insertsection}
	\framesubtitle{\insertsubsection}
	\begin{itemize}
		\item<1-> Асимметричные криптографические системы основаны на так называемых односторонних функциях с секретом.\newline 
		Например, операция возведения числа в степень по модулю: \newline 
		\begin{center} \( c\equiv f(m) \equiv m^e \pmod{n} \) \newline
		\( m\equiv f^{-1}(c) \equiv c^d \pmod{n} \) \newline 
		\( d\equiv e^{-1} \pmod{\varphi(n)} \) \newline \end{center}
		\item<2-> \boldsymbol{\alert{c}} получено возведением в степень по модулю числа m. Назовём это действие шифрованием
		\newline 
		\boldsymbol{\alert{m}} - открытый текст, а \boldsymbol{\alert{c}} - шифртекст
		\newline 
		Пара чисел (\boldsymbol{\alert{e}}, \boldsymbol{\alert{n}}) - открытый ключ
		\newline
		\boldsymbol{\alert{d}} - закрытый ключ 
	\end{itemize}
\end{frame}

\subsection{Математическая модель}

\begin{frame}
	\frametitle{\insertsection}
	\framesubtitle{\insertsubsection}
    \begin{itemize}
        \item<1-> 
            Пусть $ \boldsymbol{\alert{n}} = p \cdot q $, где где p и q – некоторые разные простые числа. Для такого n функция Эйлера имеет вид: \newline 
            \begin{center} $\varphi(n) = (p - 1) \cdot (q - 1)$ \end{center}
            \vspace{3mm}
        \item<2-> 
            Выберем такое число \boldsymbol{\alert{e}}: \newline 
            \begin{center}
               \hspace{10mm} $ e \in [3, \varphi(n) - 1]$ \newline 
                $НОД(e, \varphi(n) - 1) = 1$
            \end{center}
            Вычислим \alert{\boldsymbol{d}}:  \newline 
            \begin{center}
                \( d\equiv e^{-1} \pmod{\varphi(n)} \)
            \end{center}
    \end{itemize}
\end{frame}

\begin{frame}
    \frametitle{\insertsection}
	\framesubtitle{\insertsubsection}
	\begin{itemize}
	    \item<1-> Возьмём в качестве сообщения число $m \in [1, n-1]$ \newline 
	    Чтобы зашифровать его, необходимо возвести \newline его в степень \boldsymbol{\alert{e}} по модулю \boldsymbol{\alert{n}} \newline 
	    \begin{center}
	         \( c\equiv m^{e} \pmod{n} \)
	         \vspace{3mm}
	    \end{center}
	    Отметим также, что $\boldsymbol{\alert{c}} \in [1, n-1]$, как и \boldsymbol{\alert{m}}. Расшифруем шифртекст, возведя его в степень закрытого ключа \boldsymbol{\alert{d}}:
	    
	    \begin{center}
	    \vspace{3mm}
	        \( m` \equiv c^d \pmod{n} \)
	        \vspace{3mm}
	    \end{center}
	\end{itemize}
\end{frame}


\section{Используемая литература}

\begin{frame}
    \frametitle{\insertsection}
    \begin{thebibliography}{1}
    \bibitem{2}RSA: от простых чисел до электронной подписи \newline 
    https://habr.com/ru/post/534014/
    \bibitem{2}Cracking short RSA keys \newline 
    https://stackoverflow.com/questions/4078902/cracking-short-rsa-keys
    \end{thebibliography}
\end{frame}

\end{document}